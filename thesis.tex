%%%%%%%%%%%%%%%%%%%%%%%%%%%%%%%%%%%%%%%%%%%%%%%%%%%%%%%%%%%%%%%%%%%%
%% I, the copyright holder of this work, release this work into the
%% public domain. This applies worldwide. In some countries this may
%% not be legally possible; if so: I grant anyone the right to use
%% this work for any purpose, without any conditions, unless such
%% conditions are required by law.
%%%%%%%%%%%%%%%%%%%%%%%%%%%%%%%%%%%%%%%%%%%%%%%%%%%%%%%%%%%%%%%%%%%%

\documentclass[
  digital,     %% The `digital` option enables the default options for the
               %% digital version of a document. Replace with `printed`
               %% to enable the default options for the printed version
               %% of a document.
%%  color,       %% Uncomment these lines (by removing the %% at the
%%               %% beginning) to use color in the printed version of your
%%               %% document
  oneside,     %% The `oneside` option enables one-sided typesetting,
               %% which is preferred if you are only going to submit a
               %% digital version of your thesis. Replace with `twoside`
               %% for double-sided typesetting if you are planning to
               %% also print your thesis. For double-sided typesetting,
               %% use at least 120 g/m² paper to prevent show-through.
  nosansbold,  %% The `nosansbold` option prevents the use of the
               %% sans-serif type face for bold text. Replace with
               %% `sansbold` to use sans-serif type face for bold text.
  colorbold, %% The `nocolorbold` option disables the usage of the
               %% blue color for bold text, instead using black. Replace
               %% with `colorbold` to use blue for bold text.
  lof,         %% The `lof` option prints the List of Figures. Replace
               %% with `nolof` to hide the List of Figures.
  lot,         %% The `lot` option prints the List of Tables. Replace
               %% with `nolot` to hide the List of Tables.
]{fithesis4}
%% The following section sets up the locales used in the thesis.
\usepackage[resetfonts]{cmap} %% We need to load the T2A font encoding
\usepackage[T1,T2A]{fontenc}  %% to use the Cyrillic fonts with Russian texts.
\usepackage[
  main=english, %% By using `czech` or `slovak` as the main locale
                %% instead of `english`, you can typeset the thesis
                %% in either Czech or Slovak, respectively.
  english, german, russian, czech, slovak %% The additional keys allow
]{babel}        %% foreign texts to be typeset as follows:
%%
%%   \begin{otherlanguage}{german}  ... \end{otherlanguage}
%%   \begin{otherlanguage}{russian} ... \end{otherlanguage}
%%   \begin{otherlanguage}{czech}   ... \end{otherlanguage}
%%   \begin{otherlanguage}{slovak}  ... \end{otherlanguage}
%%
%% For non-Latin scripts, it may be necessary to load additional
%% fonts:
\usepackage{paratype}
\def\textrussian#1{{\usefont{T2A}{PTSerif-TLF}{m}{rm}#1}}
%%
%% The following section sets up the metadata of the thesis.
\thesissetup{
    date        = 2023/03/08,
    university  = mu,
    faculty     = fi,
    type        = bc,
    department  = Department of Computer Science,
    author      = Jan Martinek,
    gender      = m,
    advisor     = {Mgr. Tomáš Foltýnek, Ph.D.},
    title       = {Interactive database of educational materials on academic integrity},
    TeXtitle    = {INTERACTIVE DATABASE OF EDUCATIONAL MATERIALS ON ACADEMIC INTEGRITY},
    keywords    = {PHP, framework, MVC, CodeIgniter, JavaScript, database, material, academic integrity},
    TeXkeywords = {PHP, framework, MVC, CodeIgniter, JavaScript, database, material, academic integrity},
    abstract    = {%
        The thesis aims to reimplement an existing WordPress plugin created by
        Leoš Lang into a new web application. The resulting application uses
        a PHP back-end supported by the CodeIgniter4 framework. The new application
        is customized directly for ENAI (European Network for Academic Integrity)
        in both design and functionality. It mostly matches the original
        functionality, modified by including changes requested by the ENAI
        members. Its primary focus is on improving the administration interface,
        file handling, and filtering. The application is availablefor browsing
        on the ENAI (European Network for Academic Integrity) website.
        
        \url{https://academicintegrity.eu/materials}~\cite{enai_materials}
        
        \url{https://github.com/Razzoved/educational-database-bp}~\cite{repository}
    },
    thanks      = {%
        I want to give my thanks mainly to my supervisor, Mgr. Tomáš Foltýnek, Ph.D.,
        for leading me through the topic and making time for regular consultations.
        I'd also like to thank my advisor, Ing. Dita Henek Dlabolová, Ph.D. and all the
        personnel at ENAI for discussing the requirements with me and for giving me valuable
        feedback. Finally, I am grateful to my family, who have kept my morale up
        during my studies.
    },
    bib         = sources.bib,
    %% Remove the following line to use the JVS 2018 faculty logo.
    facultyLogo = fithesis-fi,
}
\usepackage{makeidx}      %% The `makeidx` package contains
\makeindex                %% helper commands for index typesetting.
%% These additional packages are used within the document:
\usepackage{paralist} %% Compact list environments
\usepackage{amsmath}  %% Mathematics
\usepackage{amsthm}
\usepackage{amsfonts}
\usepackage{url}      %% Hyperlinks
\usepackage[fencedCode]{markdown} %% Lightweight markup
\usepackage{listings} %% Source code highlighting
\lstset{
  basicstyle      = \ttfamily,
  identifierstyle = \color{black},
  keywordstyle    = \color{blue},
  keywordstyle    = {[2]\color{cyan}},
  keywordstyle    = {[3]\color{olive}},
  stringstyle     = \color{teal},
  commentstyle    = \itshape\color{magenta},
  breaklines      = true,
}
\usepackage{floatrow} %% Putting captions above tables
\floatsetup[table]{capposition=top}
\usepackage[babel]{csquotes} %% Context-sensitive quotation marks

\newcommand{\imgref}[1]{(see Fig. \ref{fig:#1})}
\newcommand{\cref}[1]{(ref. \ref{#1})}

\begin{document}
%% The \chapter* command can be used to produce unnumbered chapters:
%% Unlike \chapter, \chapter* does not update the headings and does not
%% enter the chapter to the table of contents. If we want correct
%% headings and a table of contents entry, we must add them manually:
% \markright{\textsc{Introduction}}
% \addcontentsline{toc}{chapter}{Introduction}

\chapter*{List of used tools}
\begin{itemize}
	\item Linux - operating system used for writing the text of the thesis.

	\item Grammarly - online tool for correcting texts in english.

	\item Texmaker - tool for writing LateX documents. It was used with
	fithesis4 template.

	\item Windows - operating system, used for writing the code and running
	the development enviroment.
	
	\item Insomnia - program used for testing the requests to the server.

	\item Visual Studio Code - code editor. All of the application's code was written in it.

	\item Laragon - program capable of running a local development server.

	\item HeidiSQL - database management tool, used alongside Laragon.

	\item Mozilla Firefox, Microsoft Edge, Google Chrome - web browsers.
	Application was tested on all of them.

	\item Git - versioning tool. GitLab FI (git management software hosted by
	the faculty) was used during development, while GitHub holds the final
	product for long-term unrestricted availability.

	\item WinSCP - tool for transferring files from one computer to another.
	The application was deployed to the server with it.

	\item Visual Paradigm - tool for creating UML diagrams.
\end{itemize}

\chapter{Introduction}

Previously, it was almost impossible for people to share information globally.
However, this has changed drastically in recent decades with the rise of the
Internet. Each and every year, the number of people using it grows. Over
63\,\% of people worldwide can utilize it to convey or acquire new
information~\cite{w3_wordpress}. With the increased usage also comes
an increase in available resources. While this is great for variety, it also
sometimes creates an obstacle for anyone searching for valuable learning
materials.

European Network for Academic Integrity (ENAI) is an organization comprising
higher education institutions that strives to support its members and the public
in the field of academic integrity. An interactive database of educational
materials has been made available on the ENAI to aid in this effort and has seen
considerable success since its conception.

The thesis bases itself on the previous version of this database application,
originally created as a more general plugin for WordPress. The plugin offered
a system supporting the management of educational materials to anyone employing
WordPress. While still widely used, the ENAI personnel wanted to move from
WordPress new solution. They were already considering updating the plugins user
interface to a more accessible look, bet were restricted by the WordPresses capabilities.
The final resolution came when the plugin stopped working after a WordPress update.
Such a decision requires the reimplementation of the plugin. The rework is also used
to introduce new features, refine the existing ones, and tailor the application specifically
to the needs of ENAI.

The application uses PHP as its \mbox{server-side} language, supported by a CodeIgniter4
framework. Frameworks provide many benefits, such as speeding up development by
providing many \mbox{often-used} functions or enhancing security and maintainability.
The project is designed to be a standalone web application, a small content
management system. It must be as portable as possible, with only a few steps
needed to run on most servers that support PHP. The user side of the application,
or front end, uses HTML, CSS, and pure JavaScript under the ECS6 standard.

The application can be divided into two almost separate areas. First is accessible by anyone
and features a view of single or all materials. It allows users to download attached files and
view related subjects and websites. The search supports suggestions and, along with filtering,
should be easy to grasp and control. The second is a restricted management area. It allows easy
and intuitive management of materials. The management includes the creation, updating, and
deletion of materials, tags, accounts, and files. A dashboard is included to allow viewing of
statistics of use.

\section{Thesis structure}

The second chapter takes a look at the overall context. The third chapter
focuses on the analysis of requirements. The fourth one focuses on design decisions,
showcasing select diagrams. The fifth chapter focuses on selected implementation
details and challenges. The sixth chapter evaluates the work and offers
additional insights and possible future improvements.

\chapter{Prerequisites}

The application is revolves around e-learning in the area of academic integrity.
It is a content management system many people use to educate themselves in this area.
Aside from personnal education, many of the available materials are intended as a tool
to support the teachers in their lessons. With this being the case, I think it important to
explain those terms.

\section{European Network for Academic Integrity}

“The European Network for Academic Integrity (ENAI) is an association gathering
educational institutions and individuals interested in maintaining and promoting
academic integrity”~\cite{enai_about}. It aims to support its members and
the public in the field of academic integrity, being the most extensive academic
integrity network in Europe. Aside from organizing the European Conference on Ethics
and Integrity in Academia, ENAI also provides several free services to the public,
such as a glossary on academic integrity, monthly webinars, self-assessment tools,
support for the victims of academic misconduct or a database of educational materials.

The mentioned database is one of the key players in supporting its efforts.
The database runs on a WordPress plugin~\cite{lang18} made by Leoš Lang.
Its caretakers wish to have it remade into a separate web application. Together
with this request, they want to change or remove several old features. Those
conditions are examined later in section~\ref{chap:analysis}.

\section{E-learning}

The first appearance of \mbox{e-learning} dates to the \mbox{mid-1960s}, but
the real growth began much later with the growth of the Internet in the late
90s~\cite{research_e-learning}. Since then, many forms of \mbox{e-learning}
have come into existence, be it interactive forms like videos or more passive
ones like \mbox{e-books}.

Most of the forms can be described by the following definition:  “E-learning is the use
of electronic technologies to create learning experiences”~\parencite[page.~1]{horton11}.
Learning is experienced with the usage of learning objects. “A learning object is a chunk
of electronic content that can be accessed individually and that ultimately accomplishes
a single learning objective and can prove it”~\parencite[page.~47]{horton11}.

\section{Content management systems}

“A content management system (CMS) is primarily a repository for multimedia content
objects”~\parencite[page.~17]{mauthe04}. According to this definition, three main
areas every CMS should provide can be identified:

\begin{itemize}
    \item collection - acquiring information into the system, be it from existing
    or newly created sources.
    \item management - modifying or removing information that has already been
    collected and saved into the system.
    \item publishing - making the content available to the users through the web,
    e-mail, or any other means, such as printed copies.
\end{itemize}

For this thesis, it is important to mention two concrete kinds of CMS. First, an aim
of the thesis. A system that handles educational materials directly and simplifies
the collection and management operations for the user. The other one is more advanced
but serves a different purpose. It allows any user to create and manage websites
without prior programming knowledge. One of these is WordPress, mentioned in
section~\cref{sect:wordpress}.

\section{PHP}

Hypertext PreProcessor (PHP) is the currently most used \mbox{server-side} scripting
language. Being an \mbox{open-source} project able to run on most operating systems
ensures portability to almost any server. Its main advantages are platform and database
independence or easy understandability with \mbox{C-like} syntax. It offers many
library functions by default, such as array or hashing helpers, easing the burden
on developers. PHP code can be embedded into HTML files. The server reads and
alters the contents based on the embedded code before sending the page to the
user~\parencite[page.~30]{welling17}. PHP is very fast and scalable. It can also service
virtually any content, whether in the form of small project or a \mbox{large-scale}
website like Facebook. The updates to PHP are mostly backward compatible, so little
or no changes are usually required. Though be warned that this is not a always the case.

\section{WordPress}
\label{sect:wordpress}

WordPress is the most popular content management system. It is an \mbox{open-source}
platform running on PHP that allows anyone to contribute to its improvement.
Due to the rapid development and wide range of downloadable plugins and themes,
WordPress maintains a high following. It is used by 63.3\,\% of all websites whose
content management system is known or 43.2\,\% in total~\cite{w3_wordpress}.

WordPress has a large and active community of users who share knowledge and
resources related to WordPress development. The use of WordPress plugins
enabled developers to participate in this community and leverage existing
WordPress features and knowledge.

The core of WordPress is designed to be more lightweight and flexible. A WordPress
plugin is a collection of PHP scripts that can be accompanied by custom stylesheets
and templates. Their purpose is to extend the core with additional functionalities.
They can make changes to the database, show content to users, integrate into
administration, automate tasks (such as comment filtering), or function as
a separate application.

\section{Existing solutions}

This section was written with help of ChatGPT.

A manual and AI-assisted search for a freely available software that could be used as a replacement
for the plugin was conducted. It resulted in a reveal of two distinct system design trends. The following
text elaborates further on those trends, adds several examples, and mentions why none of them was
selected for for the project.

The first of the trends if a category of systems known as Learning Management Systems (LMS). Those
are a kind of CMS, albeit with a more specific focus. This focus is on delivering full-length courses. The terms
'course' and 'educational material' are often used interchangeably, but the two have a number of key differences.

A course is a structured educational experience that is designed to provide learners with a specific set of
learning outcomes or objectives. A course typically has a defined start and end date and may be delivered
in a variety of formats, such as in-person classes, online modules, or a combination of both. Courses usually
have a set curriculum outlining their topics and learning activities. They may include assessments or evaluations
to measure the learner's progress and understanding.

Educational material refers to any content or resource designed to support learning, education, both from
the view of students and teachers. Such resources include textbooks, exercises, videos, tutorials, articles, and more.
Unlike a course, educational materials are often self-paced, self-directed experiences. This means that learners
and educators can access and use them on their own schedule and at their own pace.

While both courses and educational material aim to support learning and education, courses are generally
more structured and have clear outcomes, and require more preparation work before release. On the other
hand, educational materials are more flexible and have more uses, like teaching other than learning.

The second trend were generalized CMS. This category contains all systems that are similar to WordPress.

\subsection{System list}

\begin{itemize}

	\item \textbf{Moodle}~\cite{moodle} is a very popular platform. It was created in 2002 by Martin Dougiamas,
	an educator from Australia. Nowadays, it has more than 160000 sites in no less than 240 countries. Moodle
	provides a wide range of features, including course management, assessments, collaboration tools, and more.
	It	is extensible with plugins.

	\newpage
	
	\item \textbf{Open edX}~\cite{openedx} is an open-source platform for delivering massive open online
	courses created by edX, a non-profit organization founded by Harvard and MIT. One of the key features is
	its modular architecture, which allows for easy customization and extension. It has a large and active
	community of developers and users who contribute to the platform's development and support, enhancing
	its overall functionality.

	\item \textbf{Canvas}~\cite{canvas} is a platform that includes features such as course management,
	assignments, quizzes, and discussions, as well as integration with other tools such as Google Drive, Zoom,
	and Microsoft Office. Canvas is written in Ruby programming language. Ruby is unavailable on the current
	hosting~\cite{forpsi_data}.

	\item \textbf{Omeka}~\cite{omeka} is specifically designed for creating and managing digital collections and
	exhibits. It includes features such as customizable themes, metadata management, and access control to
	create a flexible and organized environment for digital content. It is a CMS.

\end{itemize}

\subsection{Final verdict on the systems}

All of the systems fullfil their given roles, and they do it very well. However, they all mostly share the same
fault, aiming for a different objective. What is needed for the project is a system tha can manage different
resources, but not in a form of whole courses. A course is difficult to prepare, and many of the materials
that the ENAI wants to gather are not only used for learning, but for teaching too, and a teacher usually
does not use an online course as a teaching tool during one of his classes.

The general CMS are denied too. One of those systems has already been tried on the server. The original
plugin broke as a result of one of the WordPress updates, which caused many issues. As such, the overhead
and possible incompatibilities those systems introduce when their plugins or applications are not under
active maintenance solidified their denial.

\section{MVC}
\label{sect:mvc}

The Model-View-Controller (MVC) pattern is a popular design pattern used in web
development. Its main idea is to separate the application into three interconnected
components: the Model, the View, and the Controller. The MVC pattern's primary
benefit is promoting the division of responsibilities and modularity. By splitting the
application into distinct components, each meant for a specific set of tasks, developers
can focus on one component at a time. Changes to that component do not necessarily
affect the other components. This makes it easier to maintain and update the
application over time.

"The Model is the part of the system that manages all tasks related to data: validation,
session state and control, data source structure. The Model greatly reduces the
complexity of the code the developer needs to write."~\cite{gilmore09}. The Model
represents the application's data and business logic, defining how data is stored,
manipulated, and retrieved. It provides all methods for interacting with the data and
is responsible for maintaining their consistency and integrity.

The View component represents the user interface and may be thought of as a front-end
of the application. It is responsible for presenting data to the user in a visually appealing
and understandable way and allowing interactions between the user and the application.
"By separating the design of the application from the logic of the application we greatly
reduce the risk of errors showing up when the designer decides to alter the interface of
that application by changing a logo or a table"~\cite{gilmore09}.

The Controller component acts as an intermediary between the Model and the View.
"A controller accepts requests and prepares the data for a response. It is also responsible
with establishing the format of that response. The Controller interacts with the Model in
order to retrieve the needed data and generates the View"~\cite{gilmore09}. A controller
may verify or transform the user inputs before passing them to the Model and generating
a response.

\chapter{Analysis and Design}
\label{chap:analysis}

The first part looks at the existing plugin. The second discusses the requirements for the new
implementation. After requirements come the use cases, data modeling, and UX design.

\section{Plugin}
\label{sect:plugin}

The plugin that has been mentioned several times already is an extension of the WordPress
content management system. The extension serves as an interface between a user and the
servers' filesystem and database. It provides teaching and learning support for various topics
on academic integrity. It has seen widespread use since its release, containing over 250
different materials. Those materials must be preserved during the transfer to the new
implementation. It is essential to determine the plugins' content and data structure to achieve
this. The following analysis is to fulfill this purpose and, moreover, review the possibility of using
the code of the plugin instead as a base instead of starting from scratch. To get more information
about the plugin, see the thesis written by the plugin creator~\cite{lang18}.

After the analysis, it was clear that putting the original code to use would prove more difficult
than starting from zero~\cref{subsect:plugin-structure}. The application is still meant to be
a replacement, so the existing user interface and data need to be taken into account while
discovering the requirements. 

\subsection{User interface}
\label{subsect:plugin-ui}

The first step in analyzing the plugin is to familiarize ourselves with the user interface,
making the next steps easier to grasp by understanding the context. Its graphical design
is quite plain, which is in no way a bad thing. Many websites are suffering from overdesign,
and the plainness makes the plugin look clean and easy to navigate in comparison.

The plugin inherits the same navigation page as its parent site. This feature cannot be a part
of the new implementation as it aims to be a separate software. The plugin adds several 
viewable pages:

\begin{itemize}
	\item \textbf{Material details} - shows all data of a single material. This includes connected
	data such as relations or files.
	\item \textbf{Multiple materials} - the materials displayed in a list. This list is paged and can
	be filtered by both the search bar and filters. Every material shows its title, thumbnail, publish
	date, rating, short description, and views. 
	\item \textbf{Materials table} - under restricted access. It shows all materials in a paged table.
	The table supports bulk operations but lacks adequate support for searching, filtering, and sorting.
	\item \textbf{Material editor} - under restricted access. It is used for editing or the creation or
	a single material and contains input fields for materials title, sender, type, and content. Additionally,
	it allows the assignment of taxonomies, thumbnail, files, or a single link. Relations can be added
	manually. All of the additional features are hard to operate, requiring a high amount of scrolling
	and clicking. For example, the thumbnail is at the bottom right side of the editor and not easily
	accessible. The same goes for taxonomies. Files must be uploaded to the shared servers directory,
	and only then can they be added to the material, which is an arduous process.
	\item \textbf{Configuration} - under restricted access. Used for setting the theme and basic
	settings like the number of shown items per page.
\end{itemize}

Taxonomies can be managed anywhere in the plugins' administration area. It should be noted
that the materials cannot have multiple external sites linked unless added directly to the content.
Material cannot have both an external link and files, always only one of the two. The search bar
includes suggestions, but they are not very responsive.

There are several things that could be improved with support for mobile devices. Most of the
buttons appear often appear out of place. The filters use a different layout from the PC version,
are broken, and cannot be closed after they were opened. Many of the filters often belong to
a single category, and since there is no support for taxonomy hierarchies, it creates large lists that
are hard to use.

\subsection{Structure}
\label{subsect:plugin-structure}

At a glance, the plugin seems straightforward. It does not use the WordPress MVC framework, instead following a standard  plugin folder structure: \begin{itemize}
	\item \textit{css/} - contains compiled SCSS files
	\item \textit{sass/} - contains uncompiled SCSS files
	\item \textit{img/} - contains default image assets
	\item \textit{inc/} - contains all serverside scripts (PHP files)
	\item \textit{js/} - contains all javascript files
	\item \textit{e-learning-file-manager.php} - plugin entry point
	\item \textit{uninstall.php} - not implemented
\end{itemize}

The plugin uses a range of custom data types and implements their related functionalities by itself.
The code is poorly documented outside of the mentioned thesis~\cite{lang18}. The whole
implementation depends heavily on WordPress and its application interface. It supports older
versions of PHP with guaranteed support for version 5. This directly translates to incompatibilities
with newer versions. It is not actively maintained, causing additional incompatibilities with
WordPress, too. Both the high dependency and incompatibilities pitched a decision to drop the
plugin's code altogether, using only its analysis. With this, the database can finally be looked at.

Since the plugin works as an educational database, it must handle educational materials. The
materials can be decomposed into the following properties: \begin{itemize}
	\item \textit{Title} - the name of the material.
	\item \textit{Sender} - the name of the person who sent or created the material.
	\item \textit{Date} - of when the material was published.
	\item \textit{Type} - indicates the content type of the material.
	\item \textit{Tags} - categories and other metadata.
	\item \textit{Comments} - provided by users to receive feedback on materials.
	\item \textit{Rating} - provided by users to favor quality materials.
	\item \textit{Content} - can be a description of provided attachments or a complete
	learning experience.
	\item \textit{Attachments} - link to a website or files on the server.
\end{itemize}

Sender as an attribute has only ever been used during the materials' creation, never shown
otherwise, and can be removed. While shown, the type has never seen much use and should be
merged into tags.

Tags are a central part of the application. They are also known as taxonomies. While the data
structure allows hierarchization, the front-end never implemented it. Due to that, they are ordered in
a category-to-value relationship. Over the years, many tags accumulated, categories started
overflowing, and their refinement is long overdue.

The rating was originally a part of the comment system, where each comment could contain a rating
of one through five. Many malicious users took advantage of the comments to spread spam and
advertisement messages, and the comments were disabled. The rating should be a separate function
in the new version, with comments never returning.

The attachments could either be links or files. Files were saved into folders based on the year and
month of their upload. This should be changed into a model where every attachment belongs directly
to the material as per the administrator's request.

\newpage

\section{Requirements}

\subsection{Functional requirements}

Based on the analysis of the plugin and discussions with the ENAI personnel, new functional
requirements were stated. Those requirements represent a set of functionalities the application
must provide its users.

\begin{itemize}
    \item creation and management of materials, resources, tags
    \item assignment of resources and tags to a material
    \item viewing and filtering (by title and tags) of all materials
    \item supporting multiple language materials (references)
    \item rating of materials
    \item setting of the home page and about page
    \item setting of a default image
    \item gather and view statistics about material usage
\end{itemize}

\subsection{Non-functional requirements}

Non-functional requirements represent a set of conditions that the application needs to be aware of
since they can limit its operation by disallowing the use of some technologies or imposing that
specific metrics such as quality of delivery be met.

\begin{itemize}
	\item application must be compatible with Forpsi server~\cite{forpsi_data}
	\item data are saved in a MySQL database
	\item PHP 7+
	\item internet connection and web browser
\end{itemize}

The application should be mobile-centric. It should be designed for mobile users first and only after
that for computers or other devices with larger screen sizes. There should be a high level of focus on
the design of a user-friendly interface. Each action should be well understood and simple to do. This
requirement is weighed more in the management section in this case.

\section{Use case diagram}

The use case diagram is a behavioral diagram from Unified Modeling Language (UML). It is a graphical
representation of the relations between the users and their requirements on the system. A use case
diagram highlights the system's functionality from the point of view of one or many users. The user is
not necessarily a physical person and is generally called an actor. The diagram differentiates between
the actors' roles, even if the actor is only one user in reality. Roles can be grouped into role
hierarchies, in which the children inherit the use cases from their parents. A use case diagram is often
used to aid in stating the systems' design goals and communicating those intentions to clients.

\begin{figure}
	\begin{center}
		\begin{minipage}{1\textwidth}
			\includegraphics[width=\textwidth]{pictures/use-case-diagram.png}
		\end{minipage}
	\end{center}
	\caption{Use case diagram}
	\label{fig:use-case}
\end{figure}

As seen on the diagram~\imgref{use-case}, the application requires a two-level design approach: the
presentation and management layers.

The presentation layer, also called the public layer, is accessible to anyone. Its primary responsibility is
the presentation of saved materials to anyone visiting the website. In this case, there are two
different functionalities. The first is a view of multiple materials, and the second is a view of a single
material in greater detail. In the first case, the materials should be searchable by their titles and, at
the same time, filterable by their assigned tags. Every user should be able to view the top-performing
materials. The details must contain full-length content with various attachments, be it links that can
be visited or files that can be downloaded, and an interactive material rating.

The management layer is the core functionality of the application. It should be restricted by an
authentication process that every user needs to pass before being granted access. This should
protect the vulnerable database from malicious actors. This process is depicted by \textit{Login}
action of any user~\imgref{use-case}. Logged-in users can log out if they desire to do so. Any user
should be able to reset his own password.

After logging in, the administrator must be able to see usage statistics and manage the database.
Database management includes the creation, modification, and removal of all relevant data entities.
Tags should be protected from unwanted removal if they are in use. Resources should be fully
managed by the editor that handles the materials, with the only exception being unused files. Unused
files are files that were assigned to some material, lived in it for some time, and then were either
replaced or removed. All the data assignable to a material can be created during its creation,
simplifying the work significantly by removing the need for repeated operations. Administrators
should be able to create new administrators. Any administrator should be able to change the
application's home reference and the reference to the About page. Each material uses a shared
default image as a thumbnail in case of having none, and this default image should also be
configurable. This configured image should always be reversible to the default one.

\section{Data modelling with ERD}

An Entity-Relationship Diagram (ERD) is a graphical representation of data entities and their
relationships. It is commonly used to model the system's data structure when designing databases.
An ERD consists of entities, entity attributes, and relationships. Every entity represents a single data
object in the system. Sometimes there is a need to connect two entities in a many-to-many
relationship, which is impossible in a typical database. An additional connector is required to separate
this relation into a many-to-one one-to-many relation. ERD does not have a limit on the number of
data entities. However, it is good practice to try and keep them at a minimum. Each entity can have
multiple attributes. An attribute is a representation of a value of a column for a given instance of an
entity. Each attribute can have constraints imposed upon it, such as a primary key or foreign key
constraint.

The new implementation will be a standalone project and does not need to conform to WordPresses'
guidelines anymore. The only limiting factor is the requirement of compatibility with a MySQL
database driver. The final schema will be different from the plugins' original. The ERD~\imgref{erd}
displays ten separate entities.

\begin{figure}
	\begin{center}
		\begin{minipage}{1\textwidth}
			\includegraphics[width=\textwidth]{pictures/erd-diagram.png}
		\end{minipage}
	\end{center}
	\caption{Entity relationship diagram}
	\label{fig:erd}
\end{figure}

Two of the entities serve as support for the application and do not have any relations. The
\textit{migrations} entity is added in case a database migration is needed. It holds information about
successful migrations and enables their rollbacking whenever defined. The \textit{config} handles the
storage and distribution of settings configurable by administrators, like the replacement of the
aforementioned default image.

The following entities store the application data.

Materials are saved into \textit{materials}. Several redundant columns are added to them to reduce
loading times and allow for easier sorting later in the implementation. The redundant columns are
\textit{material\_views}, \textit{material\_rating} and \textit{material\_rating\_count}. Each can be
computed by aggregating their relative tables, but aggregation is costly and can mostly be done on
update instead of select. We can presume that selection is a much more frequent operation.

Initially called taxonomies, tags are stored in a \textit{properties}. Tags will also be called properties
in the implementation. Properties can be part of property hierarchies, forming tree-like structures
where \textit{property\_tag} acts as a parent of the property. An additional connector table is
required to assign properties to materials without duplication.

Most of the original metadata is contained in \textit{resources} and connected to materials by a
connector entity. Resources include URLs, downloadable files, and thumbnails. Paths are stored
relative to the root folder of the application. Links have the entire value saved.

Related materials are kept inside of \textit{material\_material} entity, where each row forms a bi
directional relation. Material can be related to an unlimited amount of other materials, but not to
self.

Each material can be rated, and the ratings are saved in \textit{ratings}. Each row holds a rating of a
material given by a single user, with the average computed and saved into the material.

The difference in the database proved hard to overcome when importing the original database to the
new implementation. It was deemed to be about the same work as manual upload, which was done
instead and served as the first real test of the system.

\section{UX Design}

"User Experience (UX) design is the act and art of crafting the interface and interactions for a website
or application~\cite{hamm14}." Wireframing is just one of its parts. It produces conceivable layouts of
what the final product could look like. Wireframes are usually bare, composed only of boxes,
triangles, and other simple shapes, but they still give the art direction. UX design is usually prefaced
with in-depth research of the target audience along with the discovery of requirements,
consideration of limitations of technologies, and, in the case of redesigning an application, the
question of what is hampering the users.

In the case of the original, most of the complaints came from the administrators, but the publicly
available view had its share of problems too. From the thesis~\cite{lang18} it can be seen that the
design and wireframing were done with larger screens in mind. In the past few years, the use of
mobile devices grew exponentially, with now over 58\,\% of worldwide internet traffic being mobile
devices~\cite{mobile_usage}. The specific issues are elaborated upon in the plugin analysis
section~\cref{sect:plugin}.

\subsection{Wireframes}

The first and foremost decision was on the layout of the filters. Filters are one of the most used parts
of the application, present in both the public and administration layers. The layout needs to support
the nesting of properties. The implemented design~\imgref{filters} needs to be very flexible. Each
\textit{property} that has children can be closed. Every property can be selected, so checkboxes have
been added. If the checkbox is checked and the item has children, the children should disappear. In
some cases, one category contained many values, so a \textit{show more/less} button is added to
further reduce the displayed item count. A similar layout is used inside material details but with no
checkboxes.

\begin{figure}[!htbp]
	\begin{center}
		\begin{minipage}{.3\textwidth}
			\includegraphics[width=\textwidth]{pictures/filters_closed.png}
		\end{minipage}
		\begin{minipage}{.3\textwidth}
			\includegraphics[width=\textwidth]{pictures/filters_opened.png}
		\end{minipage}
	\end{center}
	\caption{Wireframes of one group of filters}
	\label{fig:filters}
\end{figure}

Another step was the creation of a materials' card~\imgref{mat-card-mobile}. The card is an overview
of the materials' information and should be used when displaying multiple materials on one page. It
shows a thumbnail at the top, followed by the title of the material, and then other details, the most
important being the rating. The whole card should indicate that it is clickable. On a larger screen, the
card is more horizontal than vertical but shows the same data~\imgref{mat-card-pc}.

\begin{figure}[!htbp]
	\begin{center}
		\begin{minipage}{.35\textwidth}
			\includegraphics[width=\textwidth]{pictures/material_mobile.png}
		\end{minipage}
	\end{center}
	\caption{Material card for mobile devices}
	\label{fig:mat-card-mobile}
\end{figure}

\begin{figure}[!htbp]
	\begin{center}
		\begin{minipage}{1\textwidth}
			\includegraphics[width=\textwidth]{pictures/material_pc.png}
		\end{minipage}
	\end{center}
	\caption{Material card}
	\label{fig:mat-card-pc}
\end{figure}

With those two components, the public page can be designed. The mobile
version~\imgref{page-mobile} is stacked in one column. On top of the page, a navigation bar is
present. The navigation bar is toggleable if the links do not fit, showing them on separate rows (3).
Under navigation, a page title is added. The filters can be toggled between hidden and shown, freeing
scrolling space. A search bar is added to add the ability to search for materials. The only item left is the
cards, which can be paged.

\begin{figure}[!htbp]
	\begin{center}
		\begin{minipage}{.25\textwidth}
			\includegraphics[width=\textwidth]{pictures/page_mobile1.png}
		\end{minipage}
		\hfill
		\begin{minipage}{.25\textwidth}
			\includegraphics[width=\textwidth]{pictures/page_mobile2.png}
		\end{minipage}
		\hfill
		\begin{minipage}{.25\textwidth}
			\includegraphics[width=\textwidth]{pictures/page_mobile3.png}
		\end{minipage}
	\end{center}
	\caption{Public mobile page: (1) default, (2) opened filters, (3) opened navigation}
	\label{fig:page-mobile}
\end{figure}

\begin{figure}[!htpb]
	\begin{center}
		\begin{minipage}{.9\textwidth}
			\includegraphics[width=\textwidth]{pictures/page_pc.png}
		\end{minipage}
	\end{center}
	\caption{Public page}
	\label{fig:page-pc}
\end{figure}

The computer version uses a two-column layout \imgref{page-pc}, with one being a much smaller
sidebar where the filters are placed. The rest is similar to the mobile version, only using the space
more freely and missing the ability to hide all the filters completely. The navigation bar should fit all
links without the need for toggling, too.

The material details use the same layout as the public page \imgref{page-pc}. The only significant
difference is the header. The header contains the information seen on cards, except the description of
the material. The thumbnail prefers being on the left side, switching to full width on mobile. The go
back button should be present to allow an intuitive return to the all materials page. Content,
attachments, and other features are placed in a column below the header.

No additional wireframes were made because the administration follows the same design guidelines.
The main differences are that administration cards contain additional information, such as ids of
materials, date of last update, or current status. They also include clickable buttons enabling editing
and deletion. Other design details were decided upon during the implementation.

\chapter{Implementation}

On the client's side, the application uses HTML to present the views and CSS to style them. The CSS
was not written directly but compiled from SCSS. To add interactivity to the application, JavaScript
was either embedded directly into the HTML files or loaded independently. The applications' server
side is written entirely in PHP. To help with the implementation, a PHP framework CodeIgniter4 was
used~\cref{sect:codeigniter}.

The application was developed on a local development server with the assistance of Laragon. Laragon
is a Windows application that can run an Apache server with a MySQL database. The development
server had been set up to match the deployment server as closely as possible.

A versioning tool, git, was used to improve the development workflow. A GitLab repository on the FI
domain was used for development, but the final product resides on GitHub with revised commit
history.

The editor of choice has been Visual Studio Code. Visual Studio Code is a powerful code editor with
easy and quick extendability through extensions. It integrates the git directly into the editor. Added
extensions were used to provide syntax highlights, code suggestions, SCSS compilation, and error
lensing.

The implementation is described by way of looking at separate use cases. Every section that follows
the authentication ~\cref{sect:authentication} belongs to the protected part of the application. 

\section{CodeIgniter4}
\label{sect:codeigniter}

CodeIgniter is an Application Development Framework - a toolkit - for people who build websites
using PHP~\cite{codeigniter}. It is a small framework that claims to offer great performance. It is
aimed at development under the MVC pattern but does not enforce it. The framework can have its
core parts easily extended or completely replaced to make the system work as needed. In short,
CodeIgniter is a malleable framework that provides the tools needed while staying out of the way. It
does not require the use of the command line, making all the features available immediately with
almost no additional setup. It is well documented~\cite{codeigniter4}, has a sizable community, and is
actively maintained. At the time of development, the latest release version and the version the
application is implemented in was CodeIgniter4. 

Several other frameworks were considered before deciding on CodeIgniter, more specifically Laravel
and Symfony. Both of them are popular and excellent choices. Feature-wise, the most complete
framework out of the three is Laravel. However, not by far, and as this was my first PHP project, the
easiest-to-learn framework was selected. CodeIgniters' documentation was also the personally most
readable.

Codeigniter4 is prepared for development beneath the MVC pattern. Due to this, the project's
structure is separated into three distinct components~\cref{sect:mvc}. The majority of the code lives
inside the \textit{App} directory:

\begin{itemize}
	\item Config - all files that configure the core application's behavior.
	\item Controllers - files serving as controllers from MVC.
	\item Database - contains database migrations and seeding. Seeding was not used.
	\item Entities - classes representing data from the database. They allow for better
	type safety and simpler workflow, providing magic methods for accessing their data
	along with their typecasting.
	\item Filters - files with logic that can be applied before or after visiting any of
	the application's routes.
	\item Libraries - shared logic.
	\item Models - files that serve as models from MVC.
	\item Validation - custom rules for verifying user inputs.
	\item Views - this folder contains all of the HTML files.
\end{itemize}

The project's entry point is a file named index.php. It is located inside the public directory. The public
directory also contains assets, CSS and JavaScript files, and the temp and uploaded folders. Temp is
where all the uploaded files that were not yet assigned live. After being assigned, those files are
moved to uploads. Uploads are categorized into folders by their material ids, each material having one
such directory.

The last directory worth mentioning is writable. Writable contains all of the logs, cache, and session
files. More on the usage of the cache will be talked about later~\cref{sect:properties}.
 
\subsection{Setup and Updating}


Before deploying CodeIgniter4, it is imperative to verify that the server can run PHP 7.4 or newer. The
\textit{intl}, \textit{mbstring}, \textit{json} extensions must be enabled. The application uses the
provided Image class to manipulate thumbnails, so a \textit{imagick} must be enabled too. The
application must be configured using a \textit{.env} file in the framework's root directory or directly
inside the \textit{App/Config} folder.

The first obstacle to the creation of the application came in the form of no way of configuring a new
virtual host on the deployment machine. Virtual hosts are the preferred way of using CodeIgniter or
any web-based application. To circumvent this, a new \textit{.htaccess} file was created. This file
reroutes all traffic that does not lead to a file from the root folder to the public folder's index, the
application's runner. The new file also prevents unwanted access to the root directory. Be wary that it
will not work whenever URL rewriting is disabled.

Running the application is as simple as renaming the applications' folder to be the same as the URL
used for accessing it and modifying the env file with the desired database credentials and base
applications URL. Before deploying the application, it is recommended to change the environment to
production from development to hide debugging tools and error messages.

Sometimes it is desirable to update the framework to a newer version. Be warned that in doing so,
incompatibilities may be introduced to the application. Updating can be done with either the
composer or manually. In both cases, the changes to the \textit{App} must be done by following the
patch notes. New versions of this directory are not applied automatically and must be merged by hand
or with specialized merge tools.

\section{Models}

The very first step in the application's implementation is the creation of entities and their models. This
is an easy process due to the knowledge from the ERD model~\imgref{erd}, mostly copying the
structure but including the connectors directly in their intended entity.

Several custom methods are included in the resource and material entities. Those methods streamline
their use in the controllers and views. The resource entity's main methods are getDefaultImage(),
getURL(), and getRootPath(). The default image method retrieves the path to the default image from
the database as a new entity. The other two methods work on the property's saved data, transforming
them into the desired outcome with the URL being a valid web reference and the root path a relative
path of the resource from the projects' root directory. The material entity then provides getter
methods for its resources. Those resources are saved in a shared array, and the methods retrieve only
the specified type.

Codeigniter provides a base model that can be extended to simplify the usage of the database. Each
extension must specify the targeted table's name, its primary key, and the fields it can modify. Not all
fields must be present, which disables their modification by the model. There is one model for each of
the databases' tables.

\section{Controllers and routing}

Next came the creation of controllers. Each entity has its controller. Controllers are separated into two
namespaces, public and administration. This is merely an organizational separation. Nonetheless, it is
used to better discern their responsibilities when routing. Every public method has a route associated
with it. The routes can either point to a viewable page or be an AJAX endpoint. All routes beginning
with \textit{/admin/} are protected by an authentication filter~\cref{sect:authentication}.

The AJAX endpoints are targeted for use by the clients' javascript. It should be noted that the
application supports Cross-site Request Forgery (CSRF) protection. CSRF is an attack that forces
authenticated users to submit unwanted actions. Every submitted request must come with a CSRF
token to be protected against this attack. The server verifies the token, and the request is approved if
valid. Every submission changes the current token. Tokens can be saved as a unique input field in every
form and loaded with the view. This is fine for standard forms, but when using AJAX, each submission
invalidates the current token but does not replace it with a new one. This makes default AJAX
unusable. All AJAX routes are grouped under the \textit{/ajax/} route to prevent this behavior. The
route segment causes the application to append the current CSRF token to the AJAX response,
allowing the replacement of all tokens currently present on the page.

Each controller provides an index method. The index, in most cases, presents the user with a view
containing all accessible data of the related entity.

\section{Views}

The views component is the last part of the application, developed alongside the controllers. Views
contain all of the applications' HTML code and some of the JavaScript. The HTML is saved in
\text{.php} files and interlaced with PHP code. The framework then executes the embedded code
before handing the result to the user. Due to that, it is possible to work with the entities directly
inside the HTML. This is both more type-safe and pleasant than passing only basic data types, like
strings and integers, across the views.

The views follow the same division into the public and protected sections as seen in controllers. They
are further separated by their entities. Again, this is purely organizational. Most of the views were
created as includable components; the rest are either layout or extend some layout.

\section{Properties as filters}
\label{sect:properties}

From the very start the properties were concieved as filters. To allow their easy handling inside the
views, it is necessary to transform them from an array into a tree. This can be achieved by taking those
properties whose tag is 0. For each such property, we find all of their children. Children are properties
whose tag equals the current id. This process can be repeated until no such properties exist, meaning
we arrived at a leaf.

\begin{figure}[!htbp]
	\begin{center}
		\begin{minipage}{.8\textwidth}
			\includegraphics[width=\textwidth]{code/property-tree.png}
		\end{minipage}
	\end{center}
	\caption{Function to get all properties as a tree}
	\label{fig:property-tree}
\end{figure}

The whole process requires an large amount of queries. To reduce the strain on the database, we can
cache the properties and the resulting tree. Cache can be checked before each query asking for a
single material. To ask for the tree, a separate cache call is made when getting the tree itself. There
are, however, complications associated with caching, mainly if the cache is saved for longer periods of
time. The cache has priority over the database, meaning that changes will not be projected until the
cache is either revalidated or it expires. Every update insert, update and delete call must be
accompanied with a cache revalidation. It is not necessary to clear the whole cache. Only the changed
property and its chain of parents along with the tree file need to be cleared.

\begin{figure}[!htbp]
	\begin{center}
		\begin{minipage}{.8\textwidth}
			\includegraphics[width=\textwidth]{code/property-cache-1.png}
		\end{minipage}
		\begin{minipage}{.8\textwidth}
			\includegraphics[width=\textwidth]{code/property-cache-2.png}
		\end{minipage}
	\end{center}
	\caption{Loading and revalidation of cache in property model}
	\label{fig:property-cache}
\end{figure}

To fulfill the request to include a possibility of adding mouseover descriptions to the properties, a
native HTML tooltips solution was used. It works by adding a \textit{title} attribute to any element,
which properties can use to display their description.

\newpage

\section{Filtering materials}

With property tree is available, it is very easy to recurse over its contents and steadily build a filter
caret. The filtering module is prepared with ease of use in mind~\imgref{filters}. Each property can be
checked. If the property is a category, this action collapses its subtree and internally selects all of the
values as a valid value for matching. In other words it groups the properties in a \textit{where in} part
of the query. Properties that are not under a selected tree must all be matched. Aside from selecting
the property, its subtree can be collapsed manually. This functionality is added to reduce the
verticality of the filters.

\begin{figure}[!htbp]
	\begin{center}
		\begin{minipage}{.8\textwidth}
			\includegraphics[width=\textwidth]{code/material-filter-1.png}
		\end{minipage}
		\begin{minipage}{.8\textwidth}
			\includegraphics[width=\textwidth]{code/material-filter-2.png}
		\end{minipage}
	\end{center}
	\caption{Filtering of materials: (1) call in model (2) call in connector }
	\label{fig:material-filter}
\end{figure}

In addition to using properties, the materials can be filtered by their titles. The search is case
insensitive, but not accent-insensitive. To improve upon it this, an accent-insensitive suggestion system
was added. The system uses a list of all available materials that is loaded with page. Search
suggestions are often done by fetching the results directly from the server. However it was observed
that such implementation seems unresponsive on the deployment server, sometimes taking seconds
to display any results, which is why this alternative was chosen. If it proves a problem in the future, it
can be changed fairly easily by replacing result fetching function with a server query.

\section{Authentication}
\label{sect:authentication}

The administration is protected by a password based authentication. This authentication uses
CodeIgniters' sessions and filters to keep users logged in. All of the protected routes begin with
\textit{/admin}. When visiting such a route, a filter is applied to verify if the user is logged in by
consulting the currently active session. This means no authentication is required until the user logged
out or the session was invalidated. 

\begin{figure}[!htbp]
	\begin{center}
		\begin{minipage}{.8\textwidth}
			\includegraphics[width=\textwidth]{code/auth-guard.png}
		\end{minipage}
	\end{center}
	\caption{Administration protection}
	\label{fig:auth-guard}
\end{figure}

To prevent users from returning to login page when they are already logged in, another filter is
applied. This filter automatically redirects them to the landing page of the administration section
upon encountering a logged in user.

\begin{figure}[!htbp]
	\begin{center}
		\begin{minipage}{.4\textwidth}
			\includegraphics[width=\textwidth]{code/auth-login-view.png}
		\end{minipage}
		\begin{minipage}{.4\textwidth}
			\includegraphics[width=\textwidth]{code/auth-reset-view.png}
		\end{minipage}
	\end{center}
	\caption{Login design and password reset design}
	\label{fig:auth-views}
\end{figure}

The login process starts with the validation of the inputted email and password. The email is first used
to find a single user. If found then tries to verify the saved hashed password with the inputted one. If
the user is not found or the password not matched, returns to the login page and displays a message
for the failed step. If the validation was successfull, saves the user inside the session and marks him as
logged in, continuing to administration. When logging out, the saved session data are destroyed.

\begin{figure}[!htbp]
	\begin{center}
		\begin{minipage}{.9\textwidth}
			\includegraphics[width=\textwidth]{code/auth-process.png}
		\end{minipage}
	\end{center}
	\caption{Authentication controller login and logout}
	\label{fig:auth-process}
\end{figure}

The application allows users to reset their password. The process consists of sending the token to the
entered email, redirecting user to the reset password page and waiting for user to enter the token and
new password. The token is checked with the token saved into the session, and if valid, sets the
password to the new one.

\section{Statistics}

The application saves usage statistics for each material. Upon visiting a material by a non-logged in
user, the materials views are incremented and control passed to the Views model. The model then
verifies if the material was visited in current server time day. If it was, then increments the count. If
not, then inserts a new record with count set to one. This enables the application to keep track of daily
views.

Another useful statistic is served by the Ratings model. Each user can have up to one rating for each
material. The model can be used to retrieve the recent most favorite materials by grouping the ratings
by update date and id of the material.

The administration uses those statistics on the dashboard page. Dashboard is used to render daily and
total visits or most viewed materials, both recent and all-time. The statistics include the blame
counter, keeping track of all last updates on materials, and their relevant users. The dashboard also
provides shortcuts to the last updated and last published materials.

\section{Creating or updating data}
\label{sect:data-management}

The administration was designed to be as resuable as possible. To that end, all of the manageable data
use the same tabular layout. This layout is a redesign of the public one for viewing all materials. To
allow easier management, each of the displayed entities can be created, edited or deleted directly in
the view. The only exception to the creation and updating is the material entity, more elaborated upon
in its own section~\cref{sect:material-editor}.

For other entities, the creation and updating is done by way of modals. The modal is an HTML element
that is displayed over the current content. Each has a title specifying the current action and a content
that represents that action. This approach allows easy integration into other views.

The application's modals all use the same JavaScript code contained in the \textit{modal.js} file. Every
modal is opened by using the modalOpen~\imgref{modal-open} function. The function transforms the
modal template from string to a HTMLElement. If the string contains scripts, it is necessary to reapply
them, since the conversion marks them as disabled and does not run them. The funtion also fetches
the data and fills the template if requested. The filling is done to allow the use of the function for
both creation and editing.

\begin{figure}
	\begin{center}
		\begin{minipage}{.8\textwidth}
			\includegraphics[width=\textwidth]{code/modal-open.png}
		\end{minipage}
	\end{center}
	\caption{JavaScript function for opening modals}
	\label{fig:modal-open}
\end{figure}

Submission of the modal is handled by modalSubmit, a function that is an abstraction of sending the
form data as a asynchronous request to the servers' controllers. The controllers then receives the
request, verifies its contents and processes it. Then it sends either an error or the result of the
operation back to the client. In case of failure the error message is displayed in the modal, otherwise
the modal is closed and the view updated with the response.

Every administrator is allowed to modify, create, and delete other users~\imgref{modal-user}. An
administrator cannot delete his own account. To prevent the unwanted modification of password
while editing an account, a checkbox is added. This checkbox enables password format verification and
adds the password to the database call. The password inputs are always present in case of new
account.

\begin{figure}[!htbp]
	\begin{center}
		\begin{minipage}{.4\textwidth}
			\includegraphics[width=\textwidth]{code/modal-user-hide.png}
		\end{minipage}
		\begin{minipage}{.4\textwidth}
			\includegraphics[width=\textwidth]{code/modal-user-show.png}
		\end{minipage}
	\end{center}
	\caption{User editor modal: (1) password hidden, (2) password shown}
	\label{fig:modal-user}
\end{figure}

A property can belong to a category. To achiveve a user friendly management, a list of categories is
loaded when opening the property editor~\imgref{modal-tag}. The category input field can then
either be empty or one of the listed categories, which is verified on each change of the field. The form
also supports the addition of description or priority.

\begin{figure}[!htbp]
	\begin{center}
		\begin{minipage}{.4\textwidth}
			\includegraphics[width=\textwidth]{code/modal-tag.png}
		\end{minipage}
	\end{center}
	\caption{Property editor modal}
	\label{fig:modal-tag}
\end{figure}

\section{Creating or updating material}
\label{sect:material-editor}

Material management is the most important feature of the application, but also the most complex. As
such, it does not use a modal as its front-end view, it is a separate page. This page contains editable
title, status, properties, content, thumbnail and files, links, and relations. It uses a third-party text
editor~\cite{tinymce} for the editing of content. The relations can be added by selecting an existing
material from a loaded list of materials. Relations are implemented bi-directionally. Adding a relation
will display it in both materials, and removing it will remove it from both, too.

Properties can be assigned to the material using a property selector~\imgref{tag-selector}. Property
selector by default shows all currently assigned properties, but it can be unlocked to allow
modifications. When the user unlocks it, the selector shows all properties of a single category, which
can be toggled on and off. It shows a single category to prevent information overload, other
categories are hidden. To change to a different category, a previous and next controls are added.

\begin{figure}[!htbp]
	\begin{center}
		\begin{minipage}{.6\textwidth}
			\includegraphics[width=\textwidth]{code/tag-selector.png}
		\end{minipage}
	\end{center}
	\caption{Unlocked property selector}
	\label{fig:tag-selector}
\end{figure}

The editor is controlled by a specialized controller. This controller handles the verification of inputs,
their transformation from post data into material entity and the follow up saving. The saving is an
extensive operation. The first step is the saving of material into the database. This is enclosed in a
transaction, and rollbacked should any part fail. During the transaction, the materials properties and
relations are saved too. After the transaction is complete, the newly unused resources, given by a
special post array, are unassigned. All operations are done by a specialized library that handles all
resource specific operations. In case of the resource being a file or thumbnail the unassignement is
different for temporary and assigned resources. The temporary ones are deleted completely, while
the assigned ones are removed from database and moved into a separate folder. In case of failure an
error message is returned and added to the editor view.

\begin{figure}[!htbp]
	\begin{center}
		\begin{minipage}{.9\textwidth}
			\includegraphics[width=\textwidth]{code/material-save.png}
		\end{minipage}
	\end{center}
	\caption{Function for saving materials from controller}
	\label{fig:material-save}
\end{figure}

\begin{figure}[!htbp]
	\begin{center}
		\begin{minipage}{.8\textwidth}
			\includegraphics[width=\textwidth]{code/material-save-db.png}
		\end{minipage}
	\end{center}
	\caption{Transaction for saving material}
	\label{fig:material-save-db}
\end{figure}

From the previous description, it can be seen that the material saving does not handle the uploading
of files, but operates only on already uploaded ones. Files, including thumbnails, are uploaded directly
on the material editor page by using asynchronous JavaScript. The uploads happen everytime a user
opens the file selector and submits it, after which the file is passed to the controller, verified and the
library stores it in the temporary directory. The client then receives the response in form of a new
resource. The resource is then added to the page, either replacing thumbnail or appended to the list
of attached files.

\section{Deletion of data}

The deletion interface is shared between all entities. Each deletable element contains a deletion
button. The button shows a confirmation modal upon interaction~\imgref{delete-modal}. This modal
is shared for all items in the view. Due to that the id of the item to be deleted is loaded into the modal
after each such interaction. The deletion action can be set to the model during this time too, however
in the application's case, it is always hardcoded into the view. Each entity has its own deletion method
inside its own controller. Those endpoints usually call the delete method of the model, however there
are two special cases.

\begin{figure}[!htbp]
	\begin{center}
		\begin{minipage}{.6\textwidth}
			\includegraphics[width=\textwidth]{code/delete-modal.png}
		\end{minipage}
	\end{center}
	\caption{Modal used for deletion}
	\label{fig:delete-modal}
\end{figure}

First is the deletion of properties. There are two condition imposed. If the property is a category,
meaning it has at least one subproperty, or if it is used in any material, it cannot be deleted and the
user is notified with corresponding error message. The verification is done by way of overriding the
property models default deletion method~\imgref{delete-tag}.

\begin{figure}[!htbp]
	\begin{center}
		\begin{minipage}{.8\textwidth}
			\includegraphics[width=\textwidth]{code/delete-tag.png}
		\end{minipage}
	\end{center}
	\caption{Deletion process of properties}
	\label{fig:delete-tag}
\end{figure}

Second is the deletion of materials. Each material is connected to a multitude of data, however the
database uses foreign keys to service its deletion, handling all of the connections inside the database
itself. Even with that being the case, there is a need to remove the attached files and thumbnail
manually. This step is done before the removal of the material from the database.

\section{Testing}

The application's testing was done manually. No unit or other tests were written.
While it is usually preferred to automatize the testing, it was not prioritized as
the creation of tests is a time-consuming process, and their task could be mostly
covered by the follow-up user testing. Extensive user, both public and administration,
 testing was done on both the pre-release and the release versions.

A randomized bot user test was considered for a while but not used. It can be presumed
that administrators will not act maliciously in the administration section. If the application
breaks due to random user behavior in the public section, it will not cause any real
issues, being recoverable by a simple refresh of the page.

\chapter{Conclusion}

The thesis aimed to analyze, design and reimplement an educational materials database WordPress
plugin into a standalone web application explicitly tailored to the needs of ENAI.

The introduction first elaborated on the context of the project, helping readers get into the problem.
The next part looked at the existence of premade systems that would be possible to use in place of
the new implementation. However, they were either too focused on the management of complete
courses, were general content management systems, or handled Wikipedias, none of which fulfilled
the application's intended functionality.

With the knowledge from the introduction, an analysis of the plugin was made. The analysis focused
on finding the exact functionality of the plugin, and its interface and data flow much more than on its
code. This way of research produced formal requirements for the application, which were then
visualized in a use case diagram and used in a data model. With both the use and data model known,
an abstract UI layout was constructed for most of the necessary components. This design was then
implemented.

The implementation uses a PHP framework~\cite{codeigniter4} as its back-end, greatly simplifying and
speeding up the development process. The framework operates on an MVC design pattern, which
significantly increases the maintainability and modularity of the application.

The product of the work is a new web application that provides easy-to-use management of an
educational database. While the application is thematically customized for ENAI, it is freely available
for use and modifications~\cite{repository} under the MIT license.

\section{Contributions of the work}

The work produced an open-source web application designed specifically for the presentation and
management of educational materials. 

Its primary and intended contribution is its deployment to the ENAI web server, which will run until it
once again becomes obsolete. This server is already quite known, and the new application may help it
grow further.

Systematic support of views and ratings should, in theory, help with improvements to the material's
quality. The application also supports a virtually unlimited level of tag hierarchies, only limited by the
complexity of the presentation to the user. This is a direct upgrade from the old implementation,
which was impossible. This upgrade should allow easier management of tags by grouping them
together and introduce an easier way of searching for whole categories. Another improvement is the
quicker and easier management of files, reducing the complexity of several steps into one.

The application is deployable with minimal setup. It also offers a setup guide. It being open-source
enables anyone to download it, possibly customize and then deploy it to their own server, all free of
charge. The introduction spoke about a shortage of applications of this kind, which means that their
number might increase when no entry costs are required.

\section{Possible future improvements}

The application meets the criteria stated at the start of the development. Nevertheless, there is
always a possibility for improvements.

For example, the support for multi-lingual materials is done by bi-directional relations, which could be
massively improved by implementing material collections. Those collections could be the parents of
each material with properties attached to them instead. Those collections could then share their tags
with their materials, allowing for easier management.

Another possible improvement would be to add support for more languages on the level of the whole
application. Some parts already, like validation, already support such a notion. However, only some do.
If a translation were to be provided, minor changes would be required across the whole application,
but a framework-supported localization could then be used.

As a last example, an algorithm for automatically finding similar relations could be implemented. This
operation is currently done by hand and is mixed with language support.

\section{Evaluation}

The application was created in a span of seven months with over five hundred commits recorded on
git. The main bulk of the work came in the later part of development. While I do not have a precise
measurement of the time spent on the project, it is at the very least three hundred hours. The costs
would undoubtedly accumulate was the project a paid one. The project is my first web application,
having learned all of the languages and principles during its development. The application has come
out well in most aspects.

\printbibliography[heading=bibintoc] %% Print the bibliography.

%     \makeatletter\thesis@blocks@clear\makeatother
%     \phantomsection %% Print the index and insert it into the
%     \addcontentsline{toc}{chapter}{\indexname} %% table of contents.
%     \printindex

\appendix %% Start the appendices.
\chapter{Source Code}

Full source code of the application can be found in the attachments or in the GitHub repository:

\url{https://github.com/Razzoved/educational-database-bp}.


\end{document}
